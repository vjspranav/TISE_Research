\section{Introduction}
WebAssembly (Wasm)\cite{wasm} is a web standard that defines a binary format and a corresponding assembly-like text format for executable code in web pages. The Wasm text format is designed as a portable, size-efficient, and embeddable format that runs with near-native performance and provides languages with low-level memory models such as C++ and Rust with a compilation target so that they can run on the web.
A webassembly compiler is a tool that converts a code written in a high-level language like C++ or Rust into webassembly code. This code can be run on a web browser or on a web server. \\
The problem this paper tries to look at is being able to run codes natively on browser. WebAssembly is meant to be solving this problem, but it requires a compilation done in advance before actually hosting it. \\
I plan to research and find out ways in which a webassembly compiler will be able to compile High Level language codes to webassembly on the browser itself. This would allow for anyone to run(compile/interpret) any language natively on browser without having to have it compiled in advance. As part of the research I’ll be looking at different possible ways to implement the compiler.
The main focus of this paper would be a study on such web compilers which themselves are written in webassembly. We first start with looking at the available solutions for the same. We then go on with pyodide as a base for our implementation. We then discuss the challenges that we faced in implementing a webassembly compiler on client side. We will broadly try to validate the study on WebAssembly compiler bugs\cite{bugsinwasm} \\
We will use Virtual Labs as a platform, where we test out our implementation for validaing our study. \\
We ask the following Research Questions to guide our study:
\begin{itemize}
    \item \textbf{RQ1:} How will the implementation of a webassembly compiler help in the development of web applications?
    \item \textbf{RQ2:} What are the challenges in implementing a webassembly compiler on client side?
    \item \textbf{RQ3:} How big of an impact do bugs in WebAssembly compilers have on the implementation requirement of Virtual Labs?
\end{itemize}